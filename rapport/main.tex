\documentclass{article}

\usepackage[utf8]{inputenc}
\usepackage[T1]{fontenc}
\usepackage{graphicx}
\usepackage[french]{babel}
\usepackage[top=2cm, bottom = 2cm, right = 2.3cm, left = 2.3 cm]{geometry}
\usepackage{amssymb}
\usepackage{amsmath}
\usepackage{stmaryrd}

\begin{document}

%\maketitle
\begin{titlepage}
    ~ \vfill
    \begin{center}
      \LARGE  Université de Bordeaux - Master 2\\[1.5cm]

      {\Large \bfseries \bsc{--- Algorithmique appliquée ---}}\\[0.5cm]

      \rule{\linewidth}{0.5mm}\\[0.4cm] {\Huge \bfseries Projet stratégies de défense à la RoboCup \\[0.2cm]} \rule{\linewidth}{0.5mm}\\[1.5cm] {
      \Large Adrien \bsc{Maurin}, Florian \bsc{Simba}}\\[0.5cm]

                {\large Encadré par Ludovic \bsc{Hofer}}\\ \vfill
                \includegraphics[width = 300px]{logo.jpg} \vfill
                                {\large 15 septembre 2020}
    \end{center}
\end{titlepage}

\section{Définition formelle du problème}

%parler collision, pas de discrétisation angulair ou de position de défenseurs.
%% la direction ne semble pas fonctionner, peut êtrre uniquement orthogonal ?
% goal à verifier l'utilité de la position ???
% nb pages

\subsection{Input}

\paragraph{La constante $posstep$} Il s'agit de la valeur qui indique l'écart minimal entre deux points de $\mathbb{R}^2$. Cela permet de discrétiser $\mathbb{R}^2$.

\paragraph{La constante $thetastep$} Il s'agit de la valeur minimale de l'angle que doivent former deux droites pour en pas être confondus. Cela permet de discrétiser les angles.

\paragraph{La constante $radius$} Il s'agit du rayon des cercles qui modélisent les robots.

\paragraph{Limites du terrain} Le terrain de jeu est un rectangle délimité par deux points $(x_{min}, y_{min})$ et $(x_{max}, y_{max})$. Il s'agit donc de l'ensemble de points $F$ tels que :

\begin{equation*}
F = \{ (x, y) \in \mathbb{R}^2 \ /\  x \leqslant x_{max} \wedge x \geqslant x_{min} \wedge y \leqslant y_{max} \wedge y \geqslant y_{min} \}.
\end{equation*}

\paragraph{Le terrain} Le terrain est modélisé par l'ensemble de points $G$ tel que :

\begin{equation*}
G = \{ (x, y) \in F \ /\  fmod(x, posstep) = 0 \wedge fmod(y, posstep) = 0 \text{ avec } posstep \in \mathbb{R}^*_+ \}.
\end{equation*}

Autrement dit, il s'agit de la grille de points contenant l'origine du plan et dont la taille des cases est de $posstep$.

\paragraph{Les robots} Les robots sont modélisés par l'ensemble de points suivants avec $p_0$ le point central du robot :

\begin{equation*}
    R_{p_0} = \{ p \in \mathbb{R}^2 \ /\  ||p_0 - p||_2 \leqslant radius \}.
\end{equation*}


\paragraph{Les positions des adversaires} Il s'agit d'une liste de points $p_i$, avec $p_i \in G$. Cette liste $A$ est définie par :

\begin{equation*}
    A = \{ \forall i, j \in \llbracket 1, |A| \rrbracket, i \ne j, \neg collision(a_i, a_j) \}.
\end{equation*}

La fonction $collision$ peut être exprimée de la manière suivante :

\begin{align*}
  collision \colon &F^2 \to \{0, 1 \}\\
  &d_i, d_j \mapsto collision(d_i, d_j) = \begin{cases}
                                   0 & \text{si $|| d_i - d_j ||_2 < 2R$ avec $R$ la taille du robot} \\
                                   1 & \text{sinon.}
  \end{cases}
\end{align*}

\paragraph{Le(s) goal(s)} Il s'agit d'un ensemble de segments délimités par deux points. Un dernier paramètre permet d'indiquer le sens du goal, i.e. par quel \og côté \fg{} du segment la droite de tir doit passer. On notera ce vecteur $g$. Le segment délimitant le goal sera noté $T$.

\paragraph{Les tirs d'un attaquant} Il s'agit d'un ensemble $T_i$ de demi-droites ayant pour point d'origine celui d'un attaquant $i$ tel que :

\begin{equation*}
    T_i = \{ t_i \subset \mathbb{R}^2 \ / \ fmod(angle(t_i), thetastep) = 0 \}.
\end{equation*}

La fonction $angle$ retourne l'angle formé par la demi-droite donnée et l'axe des abscisses.

\paragraph{Les tirs cadrés}
Pour qu'un tir soit cadré, il faut que le tir d'un attaquant traverse le segment du goal dans le bon sens. Cette représentation montre les tirs qui atteignent le but sans prendre en compte les robots. L'ensemble $B_c$ des tirs cadré est exprimé ainsi :

\begin{equation*}
    B_c = \{ t_{ij} \ / \ \forall i \in \llbracket 1, |A| \rrbracket, \forall j \in \llbracket 1, |T_i| \rrbracket, t_{ij} \cap T \ne \emptyset \wedge sens(t_{ij}, a_i) \}
\end{equation*}

\begin{align*}
  sens \colon & F^2, F^2\to \{0, 1 \}\\
  &T_i, a_i \mapsto sens(T_i, a_i) = \begin{cases}
                                   1 & \parbox[t]{.7\textwidth}{si $\exists t \in T_i\ / \ \exists (x, y) \in t \textbackslash \{A_i \} \ / \ (signe(x-a_{ix}) \ne signe(g_x) \wedge g_x \ne 0) \ \vee (signe(y-a_{iy}) \ne signe(g_y) \wedge g_y \ne 0)$ avec $g$ le vecteur direction du goal} \\
                                   0 & \text{sinon.}
  \end{cases}
\end{align*}


\paragraph{Les tirs non-cadrés}
Pour qu'un tir soit non-cadré, il faut que tir d'un attaquant ne traverse pas le segment du goal, ou dans le mauvais sens sinon. Il s'agit du complémentaire de l'ensemble précédent. L'ensemble $B_{nc}$ est exprimé ainsi :

\begin{equation*}
    B_{nc} = \{ t_{ij} \ / \ \forall i \in \llbracket 1, |A| \rrbracket, \forall j \in \llbracket 1, |T_i| \rrbracket, t_{ij} \cap T = \emptyset \vee \neg sens(T_i, a_i) \}
\end{equation*}

\paragraph{Les tirs interceptés}
Pour qu'un tir soit intercepté, il doit dans un premier temps être cadré et rencontrer un obstacle sur le chemin (robot allié ou adverse). L'ensemble $B_i$ est exprimé ainsi :

\begin{equation*}
    B_i = \{  t_i \ / \ \forall i \in \llbracket 1, |B_c| \rrbracket, \exists ! j \in \llbracket 1, |A| \rrbracket, R_{a_j} \wedge t_i \ne \emptyset \wedge t_i \cap D = \emptyset \}
\end{equation*}



\paragraph{Les tirs réussis}
Pour qu'un but soit marqué, il faut qu'une droite de tir d'un attaquant traverse le segment du goal dans le bon sens sans rencontrer de robots sur le trajet. L'ensemble $B$ est exprimé ainsi :

\begin{equation*}
    B = \{t_{ij} \ / \ \forall i \in \llbracket 1, |A| \rrbracket, \forall j \in \llbracket 1, |T_i| \rrbracket, t_{ij}  \in B_c \wedge t_{ij} \notin B_i  \}
\end{equation*}



\subsection{Output}

\paragraph{Les positions des défenseurs} Il s'agit d'une liste de points $p_i$ dans $D$ tels que :

\begin{equation*}
D = \{ \forall i, j \in \llbracket 1, |D| \rrbracket, i \ne j, \neg collision(d_i, d_j) \}.
\end{equation*}

Le cardinal de l'ensemble $D$ est par ailleurs borné par le nombre de joueurs. On a ainsi $|D| \leqslant 8$.



\section{Modélisation du problème}

% parler du langage de prog ???
% a voir ce qu'il faut mettre dedans
%demander au prof !!!!!
% modéliser sous forme de graphe
% algorithme A* => solution ?

\section{Pistes de résolution du problème}

\subsection{Méthode naïve : barrière de défenseurs}

\subsection{Les défenseurs se placent devant les attaquants}
L'idée de cette stratégie consiste à positionner chaque défenseur devant un attaquant de manière à bloquer son champ de tir. Cette technique nécessite autant de défenseurs qu'il y a d'attaquants.

%TODO figure tikz explicative, ainsi que présentation des calculs

\section{Compatibilité de la modélisation choisie avec les extensions}

% expliquer les limites et avantages de chaque méthodes justifiant un choix.

\subsection{Méthode naïve : barrière de défenseurs}
%- pas très utile en jeu...

\subsection{Les défenseurs se placent devant les attaquants}
%- autant de défenseurs que d'attaquant
% - les défenseurs doivent etre proche des attaquants
% - compliqué en cas de plusierus goals

\section{Limites}

\subsection{Méthode naïve : barrière de défenseurs}
Une barrière de défenseurs fonctionne bien avec un seul goal peut importe la position des adversaires
mais commence à montrer ses limites avec deux ou plusieurs goals.
En plus, cela poserait un vrai problème dans un vrai match, car il s'agirait d'un jeu purement defensif
ce qui diminuerait les chances de marquer.

\subsection{Les défenseurs se placent devant les attaquants}
Cette stratégie est très efficace par rapport à la précedente et diminuent fortement le normbre de défenseurs et
s'adapte bien au situation complexe comme la présence de plusieurs goals ou bien si l'équipe dispose de moins de
défenseurs que d'adversaires.

\end{document}
