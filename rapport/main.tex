\documentclass{article}

\usepackage[utf8]{inputenc}
\usepackage[T1]{fontenc}
\usepackage{graphicx}
\usepackage[french]{babel}
\usepackage[top=2cm, bottom = 2cm, right = 2.3cm, left = 2.3 cm]{geometry}
\usepackage{amssymb}
\usepackage{amsmath}
\usepackage{stmaryrd}

\begin{document}

%\maketitle
\begin{titlepage}
    ~ \vfill
    \begin{center}
      \LARGE  Université de Bordeaux - Master 2\\[1.5cm]
  
      {\Large \bfseries \bsc{--- Algorithmique appliquée ---}}\\[0.5cm]
  
      \rule{\linewidth}{0.5mm}\\[0.4cm] {\Huge \bfseries Projet stratégies de défense à la RoboCup \\[0.2cm]} \rule{\linewidth}{0.5mm}\\[1.5cm] {
      \Large Adrien \bsc{Maurin}, Florian \bsc{Simba}}\\[0.5cm]
  
                {\large Encadré par Ludovic \bsc{Hofer}}\\ \vfill
                \includegraphics[width = 300px]{logo.jpg} \vfill
                                {\large 15 septembre 2020}
    \end{center}
\end{titlepage}

\section{Définition formelle du problème}

%parler collision, pas de discrétisation angulair ou de position de défenseurs.

\subsection{Input}

\paragraph{Limites du terrain} Le terrain de jeu est un rectangle délimité par deux points $(x_{min}, y_{min})$ et $(x_{max}, y_{max})$. Il s'agit donc de l'ensemble de points $F$ tels que :

\begin{equation*}
F = \{ (x, y) \in \mathbb{R}^2 \ /\  x \leqslant x_{max} \wedge x \geqslant x_{min} \wedge y \leqslant y_{max} \wedge y \geqslant y_{min} \}.
\end{equation*}

\paragraph{Le terrain} Le terrain est modélisé par l'ensemble de points $G$ tel que :

\begin{equation*}
G = \{ (x, y) \in F \ /\  fmod(x, posstep) = 0 \wedge fmod(y, posstep) = 0 \text{ avec } posstep \in \mathbb{R} \}.
\end{equation*}

Autrement dit, il s'agit de la grille de points contenant l'origine du plan et dont la taille des cases est de $posstep$.

\paragraph{Les positions des adversaires} Il s'agit d'une liste de points $p_i$, avec $p_i \in G$. Cette liste $A_i$ est définie par :

\begin{equation}
    A = \{ \forall i, j \in \llbracket 1, |A| \rrbracket, i \ne j, \neg collision(a_i, a_j) \}.
\end{equation}

\paragraph{Les tirs} Il s'agit d'un ensemble $T$ de demi-droites ayant pour point d'origine celui d'un attaquant $i$ tel que :

\begin{equation*}
    T_i = \{ p \in \mathbb{R}^2 \ / \ fmod(angle(p), thetastep) = 0 \}.
\end{equation*}

La fonction $angle$ retourne l'angle formé par la droite passant par un point donné et l'axe des abscisses.

\paragraph{Le(s) goal(s)} Il s'agit d'un segment délimité par deux points. Un dernier paramètre permet d'indiquer le sens du goal, i.e. par quel \og côté \fg{} du segment la droite de tir doit passer. On notera ce vecteur $g$. Le segment délimitant le goal sera noté $T$

\paragraph{Les buts} Pour qu'un but $B$ soit marqué, il faut qu'une droite de tir d'un attaquant traverse le segment du goal dans le bon sens sans rencontrer de robots sur le trajet.

\begin{equation}
    B = \{ \forall i \in \llbracket 1, |A| \rrbracket, T_i \cap (A \textbackslash \{a_i\} \cup D) = \emptyset \wedge T_i \cap T \ne \emptyset \wedge sens(T_i, a_i) \}
\end{equation}

\begin{align*}
  sens \colon & F^2, F^2\to \{0, 1 \}\\
  &T_i, a_i \mapsto sens(T_i, a_i) = \begin{cases}
                                   1 & \parbox[t]{.7\textwidth}{si $\exists t \in T_i\ / \ \exists (x, y) \in t \textbackslash \{A_i \} \ / \ (signe(x-a_{ix}) \ne signe(g_x) \wedge g_x \ne 0) \ \vee (signe(y-a_{iy}) \ne signe(g_y) \wedge g_y \ne 0)$ avec $g$ le vecteur direction du goal} \\
                                   0 & \text{sinon.}
  \end{cases}
\end{align*}




\begin{equation*}
    R = \{ p \in \mathbb{R}^2 \ /\  ||p_0 - p||^2 \leqslant radius \}.
\end{equation*}

\paragraph{positon step}

\paragraph{theta step}

\subsection{Output}

\paragraph{Les positions des défenseurs} Il s'agit d'une liste de points $p_i$ dans $D$ tels que :

\begin{equation*}
D = \{ \forall i, j \in \llbracket 1, |D| \rrbracket, i \ne j, \neg collision(d_i, d_j) \}.
\end{equation*}

Le cardinal de l'ensemble $D$ est par ailleurs borné par le nombre de joueurs. On a ainsi $|D| \leqslant 8$.

La fonction $collision$ peut être exprimée de la manière suivante :

\begin{align*}
  collision \colon &F^2 \to \{0, 1 \}\\
  &d_i, d_j \mapsto collision(d_i, d_j) = \begin{cases}
                                   0 & \text{si $|| d_i - d_j ||_2 < 2R$ avec $R$ la taille du robot} \\
                                   1 & \text{sinon.}
  \end{cases}
\end{align*}



\paragraph{Le nombre de défenseurs}


\section{Modélisation du problème}

% parler du langage de prog ???
% a voir ce qu'il faut mettre dedans

\section{Pistes de résolution du problème}

\subsection{Méthode naïve : barrière de défenseurs}

\subsection{Les défenseurs se placent devant les attaquants}
L'idée de cette stratégie consiste à positionner chaque défenseur devant un attaquant de manière à bloquer son champ de tir. Cette technique nécessite autant de défenseurs qu'il y a d'attaquants. 

%TODO figure tikz explicative, ainsi que présentation des calculs

\section{Compatibilité de la modélisation choisie avec les extensions}

% expliquer les limites et avantages de chaque méthodes justifiant un choix.

\subsection{Méthode naïve : barrière de défenseurs}
%- pas très utile en jeu...

\subsection{Les défenseurs se placent devant les attaquants}
%- autant de défenseurs que d'attaquant
% - les défenseurs doivent etre proche des attaquants
% - compliqué en cas de plusierus goals



\end{document}
