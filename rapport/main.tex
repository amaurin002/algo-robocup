\documentclass{article}

\usepackage[utf8]{inputenc}
\usepackage[T1]{fontenc}
\usepackage{graphicx}
\usepackage[french]{babel}
\usepackage[top=2cm, bottom = 2cm, right = 2.3cm, left = 2.3 cm]{geometry}
\usepackage{amssymb}
\usepackage{amsmath}
\usepackage{stmaryrd}

\begin{document}

%\maketitle
\begin{titlepage}
    ~ \vfill
    \begin{center}
      \LARGE  Université de Bordeaux - Master 2\\[1.5cm]

      {\Large \bfseries \bsc{--- Algorithmique appliquée ---}}\\[0.5cm]

      \rule{\linewidth}{0.5mm}\\[0.4cm] {\Huge \bfseries Projet stratégies de défense à la RoboCup \\[0.2cm]} \rule{\linewidth}{0.5mm}\\[1.5cm] {
      \Large Adrien \bsc{Maurin}, Florian \bsc{Simba}}\\[0.5cm]

                {\large Encadré par Ludovic \bsc{Hofer}}\\ \vfill
                \includegraphics[width = 300px]{logo.jpg} \vfill
                                {\large 15 septembre 2020}
    \end{center}
\end{titlepage}

\section{Définition formelle du problème}

%parler collision, pas de discrétisation angulair ou de position de défenseurs.
%% la direction ne semble pas fonctionner, peut êtrre uniquement orthogonal ?
% goal à verifier l'utilité de la position ???
% nb pages

\subsection{Input}

\paragraph{La constante $posstep$} Il s'agit de la valeur qui indique l'écart minimal entre deux points de $\mathbb{R}^2$. Cela permet de discrétiser $\mathbb{R}^2$.

\paragraph{La constante $thetastep$} Il s'agit de la valeur minimale de l'angle que doivent former deux droites pour ne pas être confondus. Cela permet de discrétiser les angles.

\paragraph{La constante $radius$} Il s'agit du rayon des cercles qui modélisent les robots.

\paragraph{Limites du terrain} Le terrain de jeu est un rectangle délimité par deux points $(x_{min}, y_{min})$ et $(x_{max}, y_{max})$. Il s'agit donc de l'ensemble de points $F$ tels que :

\begin{equation*}
F = \{ (x, y) \in \mathbb{R}^2 \ /\  x \leqslant x_{max} \wedge x \geqslant x_{min} \wedge y \leqslant y_{max} \wedge y \geqslant y_{min} \}.
\end{equation*}

\paragraph{Le terrain} Le terrain est modélisé par l'ensemble de points $G$ tel que :

\begin{equation*}
G = \{ (x, y) \in F \ /\  fmod(x, posstep) = 0 \wedge fmod(y, posstep) = 0 \text{ avec } posstep \in \mathbb{R}^*_+ \}.
\end{equation*}

Autrement dit, il s'agit de la grille de points contenant l'origine du plan et dont la taille des cases est de $posstep$.

\paragraph{Les robots} Les robots sont modélisés par l'ensemble de points suivants avec $p_0$ le point central du robot :

\begin{equation*}
    R_{p_0} = \{ p \in \mathbb{R}^2 \ /\  ||p_0 - p||_2 \leqslant radius \}.
\end{equation*}


\paragraph{Les positions des adversaires} Il s'agit d'une liste de points $p_i$, avec $p_i \in G$. Cette liste $A$ est définie par :

\begin{equation*}
    A = \{ \forall i, j \in \llbracket 1, |A| \rrbracket, i \ne j, \neg collision(a_i, a_j) \}.
\end{equation*}

La fonction $collision$ peut être exprimée de la manière suivante :

\begin{align*}
  collision \colon &F^2 \to \{0, 1 \}\\
  &d_i, d_j \mapsto collision(d_i, d_j) = \begin{cases}
                                   0 & \text{si $|| d_i - d_j ||_2 < 2radius$ } \\
                                   1 & \text{sinon.}
  \end{cases}
\end{align*}

\paragraph{Le(s) goal(s)} Il s'agit d'un ensemble de segments délimités par deux points. Un dernier paramètre permet d'indiquer le sens du goal, i.e. par quel \og côté \fg{} du segment la droite de tir doit passer. On notera ce vecteur $g$. Le segment délimitant le goal sera noté $T$.

\paragraph{Les tirs d'un attaquant} Il s'agit d'un ensemble $T_i$ de demi-droites ayant pour point d'origine celui d'un attaquant $i$ tel que :

\begin{equation*}
    T_i = \{ t_i \subset \mathbb{R}^2 \ / \ fmod(angle(t_i), thetastep) = 0 \}.
\end{equation*}

La fonction $angle$ retourne l'angle formé par la demi-droite donnée et l'axe des abscisses.

\paragraph{Les tirs cadrés}
Pour qu'un tir soit cadré, il faut que le tir d'un attaquant traverse le segment du goal dans le bon sens. Cette représentation montre les tirs qui atteignent le but sans prendre en compte les défenseurs. L'ensemble $B_c$ des tirs cadré est exprimé ainsi :

\begin{equation*}
    B_c = \{ t_{ij} \ / \ \forall i \in \llbracket 1, |A| \rrbracket, \forall j \in \llbracket 1, |T_i| \rrbracket, t_{ij} \cap T \ne \emptyset \wedge sens(t_{ij}, a_i) \}
\end{equation*}

\begin{align*}
  sens \colon & F^2, F^2\to \{0, 1 \}\\
  &T_i, a_i \mapsto sens(T_i, a_i) = \begin{cases}
                                   1 & \parbox[t]{.7\textwidth}{si $\exists t \in T_i\ / \ \exists (x, y) \in t \textbackslash \{A_i \} \ / \ (signe(x-a_{ix}) \ne signe(g_x) \wedge g_x \ne 0) \ \vee (signe(y-a_{iy}) \ne signe(g_y) \wedge g_y \ne 0)$ avec $g$ le vecteur direction du goal} \\
                                   0 & \text{sinon.}
  \end{cases}
\end{align*}


\paragraph{Les tirs non-cadrés}
Pour qu'un tir soit non-cadré, il faut que tir d'un attaquant ne traverse pas le segment du goal, ou dans le mauvais sens sinon. Il s'agit du complémentaire de l'ensemble précédent. L'ensemble $B_{nc}$ est exprimé ainsi :

\begin{equation*}
    B_{nc} = \{ t_{ij} \ / \ \forall i \in \llbracket 1, |A| \rrbracket, \forall j \in \llbracket 1, |T_i| \rrbracket, t_{ij} \cap T = \emptyset \vee \neg sens(T_i, a_i) \}
\end{equation*}

\paragraph{Les tirs interceptés}
Pour qu'un tir soit intercepté, il doit dans un premier temps être cadré et rencontrer un obstacle sur le chemin (robot allié ou adverse). L'ensemble $B_i$ est exprimé ainsi :

\begin{equation*}
    B_i = \{  t_i \ / \ \forall i \in \llbracket 1, |B_c| \rrbracket, \exists ! j \in \llbracket 1, |A| \rrbracket, R_{a_j} \wedge t_i \ne \emptyset \wedge t_i \cap D = \emptyset \}
\end{equation*}



\paragraph{Les tirs réussis}
Pour qu'un but soit marqué, il faut qu'une droite de tir d'un attaquant traverse le segment du goal dans le bon sens sans rencontrer de robots sur le trajet. L'ensemble $B$ est exprimé ainsi :

\begin{equation*}
    B = \{t_{ij} \ / \ \forall i \in \llbracket 1, |A| \rrbracket, \forall j \in \llbracket 1, |T_i| \rrbracket, t_{ij}  \in B_c \wedge t_{ij} \notin B_i  \}
\end{equation*}



\subsection{Output}

\paragraph{Les positions des défenseurs} Il s'agit d'une liste de points $p_i$ dans $D$ tels que :

\begin{equation*}
D = \{ \forall i, j \in \llbracket 1, |D| \rrbracket, i \ne j, \neg collision(d_i, d_j) \wedge B = \emptyset \}.
\end{equation*}

Le cardinal de l'ensemble $D$ est par ailleurs borné par le nombre de joueurs. On a ainsi $|D| \leqslant 8$.



\section{Modélisation du problème}

% parler du langage de prog ???
% a voir ce qu'il faut mettre dedans
%demander au prof !!!!!
% modéliser sous forme de graphe
% algorithme A* => solution ?

On va modéliser le problème qu'on vient de définir de manière formelle comme un problème de graphe.

\paragraph{Les n\oe uds représentant les tirs}
Un tir est un sommet du graphe combinant la position d'un attaquant et un angle $(Att, \theta)$. On ne considère que les tirs cadré. On note l'ensemble de ces sommets $V_t$.

\paragraph{Les n\oe uds représentant les position des défenseurs}
La position d'un défenseur est un sommet modélisé par un point du terrain. On ne considère que les positions disponibles. On note l'ensemble de ces sommets $V_d$.

\paragraph{Les interceptions}
Une interception est modélisée comme une arête reliant le tir d'un attaquant à une position de défenseur. On note cet ensemble $E$.

\paragraph{Simplification du graphe}
On simplifie le graphe en retirant les sommets reliés à aucune arrête (correspondant à des positions de défenseurs n'interceptant aucun tir).

\paragraph{Ensemble dominants}
Résoudre le problème initial, i.e. trouver le nombre minimal de défenseurs permettant de défendre le but se ramène dans notre modélisation à résoudre le problème de graphe suivant : il faut trouver le plus petit sous ensemble de $V_t$ tel que cet ensemble domine $V_d$.

%todo figure tikz pour modéliser le pb



%%\paragraph{Les n\oe uds du graphe}
%%Les n\oe uds du graphe sont les points de placement dans le terrain %%(correspondant à l'ensemble $G$). On note cet ensemble $V$.
%%
%%
%%\paragraph{Les n\oe uds du goal}
%%Les n\oe uds du goal sont l'ensemble des points de $E_a$ tel que :
%%
%%\begin{equation*}
%%  V_a = \{ x \ / \ \exists y \in T \wedge x \in G \wedge ||xy||_2 < posstep \}
%%\end{equation*}
%%
%%\paragraph{Les n\oe uds des attaquants}
%%Les n\oe uds des attaquants sont l'ensemble des points de $A$. On note cet %%ensemble $V_d$.
%%
%%\paragraph{Les arêtes}
%%On définit une arête entre deux n\oe uds de la manière suivante :
%%
%%\begin{equation*}
%%  E = \{ xy \ / \ x \in F, y \in F, ||xy||_2 = posstep \}
%%\end{equation*}
%%
%%On obtient ainsi une grille.

\section{Pistes de résolution du problème}
L'idée serait de modéliser un graphe de défenseurs et d'attaquant en partant de la modélisation du problème.

Pour cela, il faut commencer par la collecte de les tirs possibles (cadrés) provenant des adversaires. Sachant, que les adversaires et le goal sont des noeuds. 
Ensuite, il faudra placer des défenseurs (noeuds) sur l'ensemble de la grille puis testé afin de retirer ceux qui n'interceptent aucun tir cadré.

Enfin, les directions de ces tirs, correspondront aux arêtes reliant les adversaires et les défenseurs ou les adversaires et le goal.

%Une fois que l'on a modélisé le problème sous forme de graphe, on peut proposer plusieurs pistes de résolution. Une première piste consiste à appliquer l'algorithme $A*$.

%\subsection{Algorithme A*}
%Pour rappel, l'algorithme A* consiste, étant donné un graphe, un sommet d'arrivée et un sommet de départ à trouver un chemin (généralement le plus cours) permettant d'aller du sommet de départ au sommet d'arrivée en passant par le graphe donné.

%\paragraph{Le sommet de départ}
%Dans cette définition du problème, le sommet de départ correspond à un attaquant (appartenant à l'ensemble $E_d$).

%\paragraph{Le sommet d'arrivée}
%Concernant le sommet d'arrivée, il s'agit plutôt d'un ensemble de sommets. Pour que le chemin soit valide, il suffit d'atteindre l'un de ces sommets. L'ensemble des sommets d'arrivée correspondent à l'ensemble des sommets du goal (l'ensemble $E_a$).

%\paragraph{Calcul d'un chemin}
%L'objectif de l'algorithme est de calculer un chemin permettant d'aller d'un attaquant à un but. On ajoutera que cette distance doit être bornée par une constante $C$ que l'on définira. Si un tel chemin existe, cela signifie qu'il est possible pour un attaquant de marquer. En calculant l'ensemble des chemins permettant d'aller d'un attaquant au goal et qui ont une longueur plus petite que $C$, on a l'ensemble des tirs de l'atraquant.

%L'objectif sera ensuite de déterminer l'ensemble des n\oe uds communs à tous les chemins. Il s'agira du placement potentiel des défenseurs pour bloquer tous les tirs.

\section{Compatibilité de la modélisation choisie avec les extensions}
\paragraph{Plusieurs buts}
S'il y a plus buts, le nombre de trajectoires ne fera que s'accroître.


\paragraph{Position initiale des joueurs}


\paragraph{Distance minimale entre les robots}
Cette extension ne va pas changer grand sur la résolution du problème, car on va considérer par exemple que cette distance minimale correspond au diamètre du robot.

\paragraph{Gardien}


\paragraph{Position initiale des joueurs}


\paragraph{Position initiale des joueurs}



\section{Limites}

\subsection{Méthode naïve : barrière de défenseurs}
Une barrière de défenseurs fonctionne bien avec un seul goal peut importe la position des adversaires
mais commence à montrer ses limites avec deux ou plusieurs goals.
En plus, cela poserait un vrai problème dans un vrai match, car il s'agirait d'un jeu purement defensif
ce qui diminuerait les chances de marquer.

\subsection{Les défenseurs se placent devant les attaquants}
Cette stratégie est très efficace par rapport à la précedente et diminuent fortement le normbre de défenseurs et
s'adapte bien au situation complexe comme la présence de plusieurs goals ou bien si l'équipe dispose de moins de
défenseurs que d'adversaires.

\end{document}
